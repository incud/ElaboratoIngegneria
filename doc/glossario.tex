\makeglossaries

\newglossaryentry{cliente}{
    name = cliente,
    description={utilizzatore del sistema e potenziale acquirente del negozio}
}
\newglossaryentry{catalogo}{
    name={catalogo},
    description={pagina che visualizza la lista di tutti i prodotti}
}
\newglossaryentry{pagina}{
    name=pagina,
    description={è il contenuto della finestra principale del software}
}
\newglossaryentry{prodotto}{
    name=prodotto,
    description={è uno degli oggetti in vendita nel negozio, quindi un disco}
}
\newglossaryentry{carrello}{
    name={carrello},
    description={pagina che visualizza i prodotti che il cliente ha segnato da acquistare, da questa pagina il cliente può finalizzare l'acquisto}
}
\newglossaryentry{utente}{
    name={utente},
    description={utilizzatore del sistema, a seconda del contesto prende il significato di cliente oppure personale}
}
\newglossaryentry{personale}{
    name={utente},
    description={utilizzatore del sistema addetto alla manutenzione del negozio}
}
\newglossaryentry{messaggio}{
    name={messaggio},
    description={sinonimo di avviso, rappresenta l'apertura di una finestra contenente un messaggio per l'utente.}
}

% usa \gls{catalogo}
% per aggiungere catalogo al glossario